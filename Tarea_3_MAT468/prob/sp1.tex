\subsection*{Solución Problema 1:}
Utilizaremos la libreria \texttt{heavy} para generar las muestral aleatorias desde una normal bi-variada.
    \begin{tcolorbox}[breakable, size=fbox, boxrule=1pt, pad at break*=1mm,colback=cellbackground, colframe=cellborder]
\begin{Verbatim}[commandchars=\\\{\}]
\PY{n+nf}{library}\PY{p}{(}\PY{l+s}{\PYZsq{}}\PY{l+s}{heavy\PYZsq{}}\PY{p}{)}
\end{Verbatim}
\end{tcolorbox}
Luego generamos un código para generar las generar las muestras aleatorias y obtener el promedio de la primera componente de los datos simulados.
  \begin{tcolorbox}[breakable, size=fbox, boxrule=1pt, pad at break*=1mm,colback=cellbackground, colframe=cellborder]
\begin{Verbatim}[commandchars=\\\{\}]
\PY{n}{alg\PYZus{}3} \PY{o}{=} \PY{n+nf}{function}\PY{p}{(}\PY{p}{)}\PY{p}{\PYZob{}}
\PY{n}{k}\PY{o}{=}\PY{l+m}{1}\PY{o}{/}\PY{l+m}{20216.335877}
\PY{n}{f} \PY{o}{=} \PY{n+nf}{function}\PY{p}{(}\PY{n}{x}\PY{p}{)} \PY{p}{(}
    \PY{n}{k}\PY{o}{*}\PY{n+nf}{exp}\PY{p}{(}
        \PY{o}{\PYZhy{}}\PY{p}{(}
            \PY{p}{(}\PY{n}{x}\PY{p}{[}\PY{l+m}{1}\PY{p}{]}\PY{o}{*}\PY{n}{x}\PY{p}{[}\PY{l+m}{2}\PY{p}{]}\PY{p}{)}\PY{o}{\PYZca{}}\PY{l+m}{2} \PY{o}{+} \PY{p}{(}\PY{n}{x}\PY{p}{[}\PY{l+m}{1}\PY{p}{]}\PY{p}{)}\PY{o}{\PYZca{}}\PY{l+m}{2} \PY{o}{+} \PY{p}{(}\PY{n}{x}\PY{p}{[}\PY{l+m}{2}\PY{p}{]}\PY{p}{)}\PY{o}{\PYZca{}}\PY{l+m}{2} \PY{o}{\PYZhy{}} \PY{l+m}{8}\PY{o}{*}\PY{n}{x}\PY{p}{[}\PY{l+m}{1}\PY{p}{]} \PY{o}{\PYZhy{}} \PY{l+m}{8}\PY{o}{*}\PY{n}{x}\PY{p}{[}\PY{l+m}{2}\PY{p}{]}
        \PY{p}{)}\PY{o}{/}\PY{l+m}{2}
    \PY{p}{)}
\PY{p}{)}
\PY{n}{t}\PY{o}{=}\PY{l+m}{1}
\PY{n}{M}\PY{o}{=}\PY{l+m}{10}\PY{o}{\PYZca{}}\PY{l+m}{5}
\PY{n}{xt} \PY{o}{=} \PY{n+nf}{c}\PY{p}{(}\PY{l+m}{0.9242179}\PY{p}{,}\PY{l+m}{2.285951}\PY{p}{)} \PY{c+c1}{\PYZsh{} x1}
\PY{n}{X1t} \PY{o}{=} \PY{n+nf}{c}\PY{p}{(}\PY{n}{xt}\PY{p}{[}\PY{l+m}{1}\PY{p}{]}\PY{p}{)}
\PY{n}{Sigma} \PY{o}{=} \PY{n+nf}{matrix}\PY{p}{(}\PY{n+nf}{c}\PY{p}{(}\PY{l+m}{1}\PY{p}{,}\PY{l+m}{0}\PY{p}{,}\PY{l+m}{0}\PY{p}{,}\PY{l+m}{1}\PY{p}{)}\PY{p}{,} \PY{n}{ncol} \PY{o}{=} \PY{l+m}{2}\PY{p}{)}
\PY{n+nf}{while }\PY{p}{(}\PY{n}{t}\PY{o}{\PYZlt{}}\PY{n}{M}\PY{p}{)}\PY{p}{\PYZob{}}
    \PY{n}{z} \PY{o}{=} \PY{n+nf}{rmnorm}\PY{p}{(}\PY{l+m}{1}\PY{p}{,}\PY{n}{Sigma}\PY{o}{=}\PY{n}{Sigma}\PY{p}{)}
    \PY{n}{y} \PY{o}{=} \PY{n}{xt} \PY{o}{+} \PY{l+m}{2}\PY{o}{*}\PY{n}{z}
    \PY{n}{p} \PY{o}{=} \PY{n+nf}{min}\PY{p}{(}\PY{n+nf}{f}\PY{p}{(}\PY{n}{y}\PY{p}{)}\PY{o}{/}\PY{n+nf}{f}\PY{p}{(}\PY{n}{xt}\PY{p}{)}\PY{p}{,}\PY{l+m}{1}\PY{p}{)}
    \PY{n}{u} \PY{o}{=} \PY{n+nf}{runif}\PY{p}{(}\PY{l+m}{1}\PY{p}{,}\PY{l+m}{0}\PY{p}{,}\PY{l+m}{1}\PY{p}{)}
    \PY{n+nf}{if }\PY{p}{(}\PY{n}{u}\PY{o}{\PYZlt{}=}\PY{n}{p}\PY{p}{)}\PY{p}{\PYZob{}}
        \PY{n}{xt} \PY{o}{=} \PY{n}{y}
    \PY{p}{\PYZcb{}} \PY{n}{else} \PY{p}{\PYZob{}}
        \PY{n}{xt} \PY{o}{=} \PY{n}{xt}
    \PY{p}{\PYZcb{}}
    \PY{n}{X1t} \PY{o}{=} \PY{n+nf}{cbind}\PY{p}{(}\PY{n}{X1t}\PY{p}{,}\PY{n}{xt}\PY{p}{[}\PY{l+m}{1}\PY{p}{]}\PY{p}{)}
    \PY{c+c1}{\PYZsh{} Agregar un paso}
    \PY{n}{t}\PY{o}{=}\PY{n}{t}\PY{l+m}{+1}
\PY{p}{\PYZcb{}}
\PY{n+nf}{return}\PY{p}{(}\PY{n+nf}{mean}\PY{p}{(}\PY{n}{X1t}\PY{p}{)}\PY{p}{)}
\PY{p}{\PYZcb{}}
\end{Verbatim}
\end{tcolorbox}
Luego corremos el algoritmo 3 veces para comparar resultados.
  \begin{tcolorbox}[breakable, size=fbox, boxrule=1pt, pad at break*=1mm,colback=cellbackground, colframe=cellborder]
\begin{Verbatim}[commandchars=\\\{\}]
\PY{n+nf}{alg\PYZus{}3}\PY{p}{(}\PY{p}{)}
\end{Verbatim}
\end{tcolorbox}
\textbf{Obteniendo el siguiente promedio}
    1.85053830658658

    
    \begin{tcolorbox}[breakable, size=fbox, boxrule=1pt, pad at break*=1mm,colback=cellbackground, colframe=cellborder]
\begin{Verbatim}[commandchars=\\\{\}]
\PY{n+nf}{alg\PYZus{}3}\PY{p}{(}\PY{p}{)}
\end{Verbatim}
\end{tcolorbox}
\textbf{Obteniendo el siguiente promedio}
    1.86571509195835

    
    \begin{tcolorbox}[breakable, size=fbox, boxrule=1pt, pad at break*=1mm,colback=cellbackground, colframe=cellborder]
\begin{Verbatim}[commandchars=\\\{\}]
\PY{n+nf}{alg\PYZus{}3}\PY{p}{(}\PY{p}{)}
\end{Verbatim}
\end{tcolorbox}
\textbf{Obteniendo el siguiente promedio}
    1.81950116666981\\
    \\
Nota que los promedios son bastante cercanos a la esperanza teórica que es aproximadamente 1.85997.